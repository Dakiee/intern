\section{Git болон GitLab}
Git бол олон хөгжүүлэгчдийг нэг кодын сан дээр зэрэг ажиллуулж, өөрчлөлтийг хянах боломжийг олгодог хяналтын систем бөгөөд, төслийн репозиторыг бий болгож, хийсэн өөрчлөлт бүрийг мессежээр баталгаажуулдаг. Кодыг хялбархан салбарлах, нэгтгэх, өмнөх хувилбарууд руу буцах боломжийг олгодог. Git нь олон хөгжүүлэгчдэд алсын зайнаас хүртэл өөрчлөлт оруулах, татах боломжийг олгодог давуу талтай.

GitLab нь Git репозиторыг удирдах, кодын хянан шалгах, асуудлыг хянах гэх мэт үйлчилгээг үзүүлдэг вэб дээр суурилсан платформ юм.

\section{PHP}
PHP (Hypertext Preprocessor) нь өргөн хэрэглэгддэг, open-source service талын програмчлалын хэл бөгөөд вэб хөгжүүлэлтэд маш тохиромжтой. Энэ нь хөгжүүлэгчдэд PHP кодыг HTML дотор оруулах замаар динамик вэб хуудас үүсгэх, мэдээллийн сантай харилцах боломжийг олгодог.

PHP ашиглахын нэг давуу тал нь платформоос хамааралгүй, өөрөөр хэлбэл Windows, Linux, Mac зэрэг янз бүрийн үйлдлийн системүүд дээр ажиллах боломжтой байдаг. Мөн PHP-д хялбархан хэрэглэх боломжтой асар их хэмжээний library, фреймворк, хөгжүүлэгчдийн томоохон нийгэмлэгтэй.

Laravel, CodeIgniter, Symphony болон бусад олон төрлийн РНР фреймворкууд нь хөгжүүлэгчдэд вэб програмуудыг бүтээх боломжийг олгодог бөгөөд энэ нь кодын ерөнхий чанарыг сайжруулж, хөгжүүлэлтийг илүү үр дүнтэй болгодог. 

\section{Laravel}
Laravel бол үнэгүй, open-source PHP вэб фреймворк бөгөөд. 
Энэ нь Model-View-Controller (MVC) архитектурын хэв маягийг ашиглан бат бөх, засвар үйлчилгээ хийх боломжтой, гоёмсог вэб программуудыг бүтээхэд хөгжүүлэгчдэд хялбар болгох зорилготой.

Laravel-ийн гол онцлогуудын нэг нь унших, ойлгоход хялбар байхаар зохион бүтээгдсэн гоёмсог синтакс нь хөгжүүлэгчдийн дунд түгээмэл сонголт болдог. Энэ нь мөн вэб програмуудыг хурдан бөгөөд үр дүнтэй бүтээхэд хялбар болгодог олон төрлийн хэрэгсэл, функцуудыг агуулдаг. Эдгээрээс дурьдвал нь:
\begin{itemize}
	\item Eloquent ORM: Laravel-ийн суулгасан Object-Relational Mapping         (ORM) хэрэгсэл нь хөгжүүлэгчдэд объект хандалтат синтакс ашиглан 
        мэдээллийн сантай харилцах боломжийг олгодог.
        \item Blade Templating Engine: Хөгжүүлэгчдэд өөрсдийн үзэл бодолд зориулж дахин ашиглах боломжтой загваруудыг үүсгэх боломжийг олгодог бөгөөд энэ нь вэб програмын бүтэц, зохион байгуулалтыг удирдахад хялбар болгодог.
        \item Artisan Command Line Interface (CLI): Laravel-ийн суурилуулсан командын мөрийн интерфейс нь загвар үүсгэх, хянагч, шилжих зэрэг нийтлэг ажлуудын багц хэрэгсэл, командуудыг хөгжүүлэгчдэд өгдөг.
        \item Route and Middleware: Laravel нь HTTP хүсэлтийг хянагч руу хүрэхээс өмнө шүүх боломжийг олгодог чиглүүлэлт болон дундын програм хангамжийг зохицуулах хялбар арга юм.
        \item Security Features: Laravel нь таны програмыг вэбийн нийтлэг эмзэг байдлаас хамгаалахын тулд шифрлэлт, нууц үгийг hash хийх, оролтын баталгаажуулалт зэрэг аюулгүй байдлын хамгаалалтын функцуудыг хангадаг.
\end{itemize}

Ларавел мөн хөгжүүлэгчдийн өргөн хүрээний, идэвхтэй нийгэмлэгтэй бөгөөд энэ нь олон тооны эх сурвалж, зааварчилгааг онлайнаар авах боломжтой байдаг.

Ерөнхийдөө Laravel нь хүчирхэг, уян хатан хүрээ бөгөөд хөгжүүлэгчдэд өндөр чанартай вэб програмуудыг хурдан бөгөөд үр дүнтэй бүтээхэд хялбар болгодог.

\section{JQuery}
jQuery нь HTML баримт бичгийг удирдах, хөдөлгөөнт дүрс үүсгэх, вэб хөгжүүлэлтэд түгээмэл тохиолддог бусад олон төрлийн ажлыг гүйцэтгэхэд хялбар болгодог хурдан, жижиг, онцлогоор баялаг JavaScript-ын libarary юм. Энэ нь вэб хуудасны элементүүдэд хандах, удирдах үйл явцыг хялбаршуулж, товч синтаксаар хангадаг.

Энд вэб хуудасны гарчгийн текстийг jQuery ашиглан хэрхэн өөрчлөх энгийн жишээг энд харуулав.

\begin{lstlisting}[language=Html, caption=JQuery ашиглаж буй байдал, frame=single]
<h1 id="heading">Hello World</h1>

<script src="https://code.jquery.com/jquery-3.6.0.min.js"></script>
<script>
    $(document).ready(function() {
        $("#heading").text("Hello jQuery");
    });
</script>
\end{lstlisting}